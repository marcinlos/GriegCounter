
\section{Moduł odczytu metadanych}
\label{sec:metadane}

\emph{Metadane} to ogół informacji o pliku dźwiękowym nie będących samymi danymi audio. W ich skład
wchodzą dane zawarte w pliku, takie jak np. tytuł utworu, autor, album, rok wykonania, a także
informacje o samym pliku: nazwa, wielkość. Te drugie nie zależą w żaden sposób od formatu pliku,
natomiast sposób przechowywania i odczytu pierwszej grupy jest częścią definicji konkretnego
formatu. Z uwagi na różnorodny, niehomogeniczny charakter metadanych, konieczne było opracowanie
sposobu ich przechowywania i udostępniania aplikacji. Niektóre metadane są kłopotliwe i czasochłonne
w obliczeniu -- np. wyznaczenie ilości próbek w pliku mp3 wymaga odczytania go w całości. Pożądanym
zatem było umożliwienie użytkownikowi frameworku określenie, które metadane należy pozyskać, a także
komunikować mu na bieżąco postępy. Rozwiązania tych problemów opisane są w tym podrozdziale.

\subsection{Sposób reprezentacji}

Metadane unikalnie identyfikowane są przez nazwę, opisującą znaczenie odpowiadających im wartości.
Wartości z kolei mogą mieć różne typy: zazwyczaj jest to łańcuch znaków, bądź liczba (najczęściej
całkowita lub stałoprzecinkowa), jednak nie można także wykluczyć innych typów, jak choćby obrazków,
reprezentujących okładkę albumu, z którego pochodzi utwór. Stąd, kolekcja metadanych stanowi
heterogeniczną mapę. Aby umożliwić przechowywanie danych różnego typu i zapewnić rozszerzalność, na
typy wartości metadanych nie nakładane są żadne ograniczenia.

Do przechowywania takiego odwzorowania wystarczyłaby mapa: \code{Map<String, Object>}. Takie
rozwiązanie ma jednak pewne wady:

\begin{itemize}

  \item brak kontroli typów -- jakkolwiek w ogólności zapewnienie statycznie pełnego bezpieczeństwa
    typów dla heterogenicznych kontenerów jest ze zrozumiałych względów niemożliwe, w wielu
    praktycznych zastosowaniach operujemy na skończonym zbiorze kluczy, którym odpowiadają znane
    typy (np. można założyć, że metadana opisująca tytuł utworu jest łańcuchem znaków), co znacząco
    poprawia sytuację

  \item niewygodny interfejs -- zwracane wartości muszą być rzutowane na odpowiedni typ, nim możliwe
    jest ich użycie.

\end{itemize}

Standardowym rozwiązaniem obydwu tych problemów dla języka Java jest opisany m. in. w
\textit{Effective Java} (rozdział 29) idiom \emph{typesafe heterogenous container}, tj. technika
polegająca na używanie informacji o typie jako części klucza. W implementacji frameworku
wykorzystana została ta idea, z zachowaniem jednak możliwości odwołania się do metadanej bez podania
jej typu.

Poniższy diagram przedtstawia strukturę klas użytą w rozwiązaniu zastosowanym we frameworku. Kolejno
opisane są wszystkie występujące na nim elementy.

\IncludeUML{metadata_storage}{Klasy przechowujące metadane}

\subsubsection{Klucz}

Klucz do heterogenicznej mapy przechowującej metadane \textit{(typ, nazwa)} reprezentowana przez
klasę \code{Key}:

\begin{java}
public final class Key<T> {

    private final Class<T> type;
    private final String name;

    // ...
}
\end{java}


Klucze odpowiadające pewnym najczęściej spotykanym i używanym metadanym zdefiniowane zostały jako
statyczne, niezmienne pola w klasie \code{AudioFeatures}:

\begin{java}
public final class AudioFeatures {

    /** Duration (in seconds) of the audio */
    public static final Key<Float> DURATION = make("duration", float.class);

    /** Author */
    public static final Key<String> AUTHOR = make("author", String.class);

    /** Title */
    public static final Key<String> TITLE = make("title", String.class);

    /** Album */
    public static final Key<String> ALBUM = make("album", String.class);

    // ...
}
\end{java}

\subsubsection{Mapa}

Klasa reprezentujaca samo odwzorowanie (mapę) składa się z kilku fragmentów. Jednym z założeń było
umożliwienie dostępu do danych zarówno przy użyciu klucza, jak i samej nazwy (ew. z typem). Aby
ułatwić korzystanie z mapy, stworzone zostały metody pomocnicze do pobierania i ustawiania wartości
dla typów podstawowywch. Zawarte są one w interfejsie \code{PrimitiveMap}. Najważniejsze, główne
metody umożliwiające dostęp i manipulowanie zawartością mapy znajdują się w interfejsie
\code{Properties}. Zapewniają one możliwość pobrania z mapy wartości na podstawie klucza
(\code{Key}), nazwy i typu metadanej (w postaci odpowiedniego obiektu \code{Class}), lub samej
nazwy, zapis przy użyciu klucza lub nazwy, a także kilka innych, pomocnych metod:

\begin{itemize}
  \item pobieranie zbioru nazw dostępnych metadanych przez \code{keySet()}
  \item pobieranie par \code{(nazwa, wartość)} przez \code{entrySet()} 
  \item konwersja do mapy nazwa $\to$ wartość przez \code{asMap()}
  \item odczytanie ilości dostępnych metadanych przez \code{size()}
  \item sprawdzenie, czy mapa jest pusta przez \code{isEmpty()}
  \item dopisanie do mapy wszystkich par z innej mapy przez \code{addAll(Properties)}
\end{itemize}

Nietrudno zauważyć, że wszystkie te operacje można efektywnie zaimplementować przy pomocy metody
\code{asMap()} i operacji z Javowego interfejsu \code{Map} (zakładając, że \code{asMap()} działa w
czasie stałym - np. zwraca mapę używaną wewnętrznie przez konkretną implementację). Stąd, kolejno
stworzona została abstrakcyjna klasa \code{AbstractProperties}, która dostarcza domyślnych
implementacji wszystkich metod interfejsu \code{Properties} prócz \code{asMap()}. Konkretną
implementacją używaną przez framework jest \code{PropertyMap}, która rozszerza
\code{AbstractProperties}, i używa wewnętrznie klasy \code{HashMap}.


\subsection{Interfejs użytkownika frameworku}

Z uwagi na szczególne wymagania dotyczące interaktywności frameworku, wynikające z celu jego
tworzenia, konieczne było zapewnienie maksymalnej ,,przezroczystości'' procesu.  Większość
metadanych jest odczytywana natychmiast, bez zauważalnego opóźnienia. Są jednak pewne wyjątki --
pliki mp3 nie zawierają \textit{explicite} zapisanej ilości próbek w utworze, aby ją poznać
konieczne jest przejście przez cały plik. W przypadku dużych utworów (np. godzinna symfonia) trwa to
stosunkowo długo, na tyle, że nie można pozwolić sobie na zwyczajne zablokowanie interfejsu do czasu
zakończenia operacji.

W związku z wymienionymi wyżej problematycznymi dla użytkownika frameworku aspektami, przy
projektowaniu interfejsu przyjęte zostały następujące wymagania:

\begin{itemize}

  \item użytkownik musi mieć możliwość określenia, które metadane są mu potrzebne, by uniknąć
    kosztownych obliczeń w przypadku, gdy nie są one konieczne

  \item użytkownik musi mieć możliwość otrzymywania na bieżąco informacji o postępach procesu, np. o
    procencie wykonania, a najlepiej także o kolejno obliczanych metadanych (zamiast czekać do końca
    przetwarzania i dopiero wówczas otrzymać wyniki, można przekazywać mu wyniki inkrementalnie)

  \item (\emph{opcjonalnie}) użytkownik powinien mieć możliwość przekazywania parametrów,
    sterujących przebiegiem procesu ekstrakcji metadanych (np. co zrobić, gdy wymagana metadana nie
    jest dostępna)

\end{itemize}

W ostatecznej wersji frameworku wszystkie 3 wymagania są spełnione (jakkolwiek ostatnia możliwość --
przekazywanie parametrów -- nie została wykorzystana w stworzonych parserach formatów). Poniższy
diagram pokazuje jego strukturę, oraz najistotniejsze elementy.

\IncludeUML{metadata_interface}
{Interfejs pobierania metadanych (pewne operacje z \code{ExtractionContext} zostały pominięte, dla
zwiększenia czytelności)}


Metoda, za pomocą której użytkownik frameworku może rozpocząć proces ekstrakcji metadanych znajduje
się w klasie \code{AudioFile}. Wydaje się to sensowne rozwiązanie, jako, że obiekt \code{AudioFile}
dysponuje wszystkimi potrzebnymi informacjami -- zna ścieżkę pliku, oraz ma dostęp do parsera, który
potrafi go zinterpretować.

Wspomniana jako jedno z wymagań komunikacja z użytkownikiem (notyfikowanie go o stanie i postępach
procesu) zrealizowana jest przy użyciu wzorca \emph{Observer} -- klient dostarcza implementacji
interfejsu, na której w odpowiednich momentach wywołane zostaną stosowne metody. Notyfikacje
podzielone zostały na dwie kategorie, przekładające się w implementacji na dwa interfejsy
listenerów:

\begin{itemize}

  \item \code{ProgressListener}, który otrzymuje wszelkie informacje o stanie ekstrakcji --
    rozpoczęciu, zakończeniu, ewentualnych błędach, szacowanym procencie wykonania

  \item \code{FeaturesListener}, który otrzymuje informacje o odczytanych metadanych (nazwa i
    wartość)

\end{itemize}

\begin{Note}
Mimo swojej nazwy, metoda \code{failed} może zostać wywołana wielokrotnie -- błąd w trakcie odczytu
metadanych nie musi implikować natychmiastowego zakończenia całego procesu.
\end{Note}


Najistotniejszą, choć pasywną, jest klasa \code{ExtractionContext}, opisująca precyzyjnie proces
ekstrakcji. Klasa ta posiada dwa zastosowania:

\begin{itemize}

  \item Umożliwia użytkownikowi konfiguracji odczytu metadanych. Używając jej metod, użytkownik może
    wybrać potrzebne mu metadane, ustawić parametry procesu, zarejestrować swoje listenery, przez
    które otrzymywać będzie informacje na bieżąco. Po udanej ekstrakcji użytkownik może także
    uzyskać dostęp do metadanych bezpośrednio przez obiekt \code{ExtractionContext}, bądź też pobrać
    z niego obiekt \code{Properties}.

  \item Umożliwia parserom, realizującym sam proces ekstrakcji, komunikowanie użytkownikowi
    istotnych faktów o jej przebiegu, oraz stanowi miejsce do przechowywania odnajdywanych
    metadanych.

\end{itemize}

Z tego powodu interfejs udostępniany przez \code{ExtractionContext} można podzielić dość wyraźnie na
dwie części: część interesującą użytkownika frameworku, i część interesującą implementatora parserów
plików audio. Szczegółowa semantyka poszczególnych metod opisana została w ich javadocu, w tym
dokumencie zostanie zatem pominięta.

\IncludeUML{metadata_interface_seq}
{Interakcje między elementami interfejsu modułu odczytu metadanych}


\subsection{Odczyt metadanych z pliku}
\label{sec:metadane_odczyt}

Jako, że format metadanych jest zależny od rodzaju pliku, ich odczytywanie jest oddelegowane do
parserów, potrafiących dany typ pliku odczytać. Udało się jednak znaleźć bibliotekę
(\emph{JAudioTagger} -- \url{http://www.jthink.net/jaudiotagger/}), która potrafi odczytywać
metadane z szeregu formatów, w tym ze wszystkich wspieranych bazowo przez framework rodzajów plików
(wav, mp3, ogg) i oferuje dostęp do nich przez jednorodny interfejs. Pozwoliło to znacząco uprościć
implementację parserów tych rodzajów plików.  Jakkolwiek niektóre wartości (np. całkowita ilość
próbek wchodzących w skład utworu) są obliczane osobno, większość pracy wykonuje biblioteka
\emph{JAudioTagger}. Poszczególne parsery używają jej poprzez nakładkę w postaci klasy
\code{JAudioTaggerMetaExtractor}. Oferuje ona prostszy interfejs, oraz tłumaczy pozyskane przez
\emph{JAudioTagger} metadane na opisany uprzednio wewnętrzny format frameworku.


\begin{java}
public class JAudioTaggerMetaExtractor {

    public JAudioTaggerMetaExtractor(File file, ExtractionContext context) { ... }

    public void run() throws IOException, DecodeException {
        try {
            AudioFile audio = AudioFileIO.read(file);
            AudioHeader header = audio.getAudioHeader();
            Tag tag = audio.getTag();

            // tlumaczenie metadanych na format frameworku
        } catch (...) {
            // tlumaczenie wyjatkow na wyjatki uzywane przez framework
        }
    }

    // ...

    public static void process(File file, ExtractionContext context) throws DecodeException, 
            IOException {
        new JAudioTaggerMetaExtractor(file, context).run();
    }

    public static void processSignalExceptions(File file, ExtractionContext context) {
        try {
            new JAudioTaggerMetaExtractor(file, context).run();
        } catch (IOException e) {
            context.failure(e);
        } catch (DecodeException e) {
            context.failure(e);
        }
    }
}
\end{java}


\code{JAudioTaggerMetaExtractor} przyjmuje w konstruktorze ścieżkę pliku dźwiękowego, na jakim ma
operować, oraz obiekt \code{ExtractionContext}, zawierający wszelkie parametry i ,,otoczenie''
obliczeń. Właściwa ekstrakcja metadanych ma miejsce w metodzie \code{run()}, gdzie wczytywany i
interpretowany jest przy użyciu klas biblioteki \emph{JAudioTagger} plik, a zdobyte informacje są
następnie przyporządkowywane odpowiednim kluczom z \code{AudioFeatures}, i przekazywane do
otrzymanej w konstruktorze instancji \code{ExtractionContext}.

Dla zwiększenia czytelności, \code{JAudioTaggerMetaExtractor} udostępnia dwie metody statyczne,
zamykające w jednym wywołaniu utworzenie obiektu i wywołanie metody \code{run()}:

\begin{itemize}

  \item \code{process} jedynie tworzy obiekt \code{JAudioTaggerMetaExtractor} i wywołuje na nim
    metodę \code{run}, wypuszczając na zewnątrz wszelkie ewentualne wyjątki

  \item \code{processSignalExceptions} robi to, co \code{process}, jednak łapie wszystkie wyjątki
    związane z ekstrakcją metadanych, i przekazuje je do metody \code{failure} obiektu
    \code{ExtractionContext}, na którym operuje. Nie są one propagowane na zewnątrz metody.

\end{itemize}

Różnica ta staje się istotna, gdy wywołanie jest jedynie częścią procesu pozyskiwania metadanych --
nawet, jeśli \code{JAudioTaggerMetaExtractor} zawiedzie, niektóre inne metadane mogą wciąż zostać
odczytane przez inne mechanizmy. 

