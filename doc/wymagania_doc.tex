\chapter{Definicje}
By móc dobrze odczytać wymagania funkcjonalne opisane poniżej będzie potrzeba przyjęcia wspólnych definicji kilku terminów:
\begin{itemize}
\item Drzewo przetwarzania – struktura drzewiasta umożliwiająca przetwarznaie i rozprowadzanie strumienia danych emitowanych przez pewne źródło. Składa się z trzech rodzajów węzłów:
\begin{itemize}
\item enumerator – źródło danych, element aktywny, korzeń (pod)drzewa przetwarzania. Posiada wyjście, pozwala na podłącznie do niego innych elementów. Produkuje dane, które następnie przekazuje podłączonym do niego węzłom.
\item iteratee – ujście danych, element pasywny, liść drzewa przetwarzania. Posiada wejście, do którego przekazywane są dane z enumeratorów, do których jest podłączony.
\item enumeratee – połączenie enumeratora i iteratee, element konsumujący strumień wejściowy i produkujący pewien strumień wyjściowy. Węzeł wewnętrzny drzewa przetwarzania.
\end{itemize}
\item Jednostka przetwarzania – element drzewa przetwarzania
\item PCM (Pulse-code modulation) – podstawowa, nieskompresowana reprezentacja dźwięku składająca się z ciągu skwantowanych wartości amplitudy fali dla kolejnych punktów w czasie, najczęściej równomiernie rozłożonych. Innymi słowy, spróbkowana fala dźwiękowa.
\item Sample enumerator – enumerator emitujący dane dźwiękowe w formacie PCM, zawierający także informację o ich formacie, oraz udostępniający podstawową kontrolę nad trwającą emisją (zatrzymanie/wznowienie).
\item Model (jako część systemu) – oparty o wzorce Observer i Composite mechanizm do propagacji wartości i notyfikacji o ich zmianach. Model ma strukturę hierarchiczną, przypomina nieco w swej konstrukcji system plików – węzły są nazwane, odwoływać się do nich można za pomocą ścieżki. Z każdego modelu w ich drzewie pozyskać można aktualną wartość, oraz pozyskiwać notyfikacje o jego zmianie.
\end{itemize}

\chapter{Przypadki użycia użytkownika końcowego}
\section{Diagramy przypadków użycia}
\IncludeUML{n_userMain}{Główne przypadki użycia użytkownika końcowego}
\IncludeUML{n_userSecond}{Wyniki analizy}
\section{Generowanie raportu analizy}
\begin{itemize}
	\item 1 Wybranie trybu pracy - użytkownik wybiera (z pomocą checkboxa), czy interesują go wykresy generowane w 	czasie rzeczywistym czy jedynie wyniki końcowe
	\item 2 Wczytanie pliku - użytkownik wskazuje plik z muzyką znajdujący się na karcie SD urządzenia
\end{itemize}
\section{Informacje na które składa się raport}
\begin{itemize}
	\item   4. Wykresy
	\begin{itemize}
		\item 4.1. Chcę zobaczyć wykres mocy fali!
		- aplikacja na żywo parsując plik generuje wykresy mocy fali (osobno dla każdego kanału)
		\item 4.2. Chcę zobaczyć wizualizację fali dźwiękowej!
		- aplikacja na żywo parsując plik generuje wykresy fali dźwiękowej (osobno dla każdego kanału), lub w przypadku opcji „bez wyników w czasie rzeczywistym” od razu pokazuje te wykresy w raporcie końcowym
		\item 4.3 Chcę zobaczyć spektrogram!
		- aplikacja na żywo parsując plik generuje spektrogram, lub w przypadku opcji „bez wyników w czasie rzeczywistym” od razu pokazuje te wykresy w raporcie końcowym

		\item 4.4 Chcę zobaczyć spektrum w danym miejscu muzyki!
		- aplikacja na żywo parsując plik generuje wykresy spektrum fali, narysowane z częstotliwością w różnych skalach (liniowej, logarytmicznej, zdyskretyzowanej logarytmicznej)
	\end{itemize}
	\item	5. Dane tekstowe
	\begin{itemize}
		\item 5.1 Chcę poznać podstawowe dane na temat pliku z muzyką!
		\begin{itemize}
			\item użytkownik wskazuje plik
			\item aplikacja, po zakończeniu analizy pokazuje również metadane takie jak długość, rozmiar, autor jeśli został wpisany
		\end{itemize}
	\end{itemize}
\end{itemize}

\chapter{Przypadku użycia użytkownika frameworku}

\section{Diagramy przypadków użycia}
\IncludeUML{n_programista}{Główne przypadki użycia programisty}
\IncludeUML{n_tree}{Przypadki użycia drzewa przetwarzania}
\IncludeUML{n_wezly}{Przypadki użycia ekstrakcji danych}
\section{Opis przypadków użycia}
\subsection{Operacje niskiego poziomu}
\begin{itemize}
	\item 1. Wczytanie pliku
	\begin{itemize}
		\item Programista dostarcza plik z muzyką
		\item Programista wywołuje odpowiednią funkcję
		\item System zwraca programiście plik w formacie PCM oraz jego metadane
	\end{itemize}
	\item 2. Dostęp do surowych danych (PCM)
	\begin{itemize}
		\item Programista potrzebuje dostępu do opisu fali dźwiękowej zawartego w pliku
		\item System oferuje dwa komplementarne podejścia:
		\begin{itemize}
			\item 2a. Model „pull” - programista odczytuje dane po kawałku do własnego bufora, ma pełną kontrolę nad przepływem sterowania w programie
			\begin{itemize}
				\item Programista wczytuje plik (1)
				\item Programista wywołuje odpowiednią metodę, zwracającą obiekt pozwalający na odczyt danych
				\item Programista tworzy bufor i przekazuje go jako argument metody tego obiektu, w wyniku czego zostaje on wypełniony kolejną porcją danych.
				\item Powyższą czynność powtarzać może aż do przeczytania całości pliku
			\end{itemize}
			\item 2b. Model „push” - programista przetwarza dane w sposób strumieniowy – dane otrzymuje przy użyciu zarejestrowanych callbacków, a kontrolę nad przepływem sterowania przekazuje na czas przetwarzania do systemu, który odczytuje zawartość pliku i przekazuje kolejne porcje danych do zarejestrowanych przez programistę listenerów. Podobny do SAX dla XML-a.
			\begin{itemize}
				\item Programista wczytuje plik (1)
				\item Programista wywołuje odpowiednią metodę, zwracającą obiekt reprezentujący strumień danych
				\item Programista rejestruje listenery
				\item Programista rozpoczyna pracę strumienia, wywołując odpowiednią metodę
			\end{itemize}
		\end{itemize}
	\end{itemize}
\end{itemize}

\subsection{Przetwarzanie pliku}
\begin{itemize}
	\item 3. Stworzenie modularnej struktury przetwarzania
	\begin{itemize}
	\item Programista projektuje wiele jednostek przetwarzających plik, niektóre z nich korzystające z wyników przetwarzania innych jednostek
	\item System dostarcza dwa mechanizmy tworzenia drzewa jednostek przetwarzania:
		\begin{itemize}
			\item bezpośrednio poprzez klasy abstrakcyjne pozwalające łączyć ze sobą węzły przetwarzania w strukturę drzewiastą
			\item poprzez klasę pomocniczą pozwalającą w deklaratywny sposób tworzyć tę strukturę używając jednostek przetwarzania identyfikowanych przez nazwy, oraz zapewniającą kontrole typów
		\end{itemize}
	\end{itemize}
	\item 4. Dynamiczne odłączanie jednostek przetwarzania.
	\begin{itemize}
		\item Twórca jednostek przetwarzania chce, by określone jednostki przetwarzania w zdefiniowanym przez niego momencie przestały przesyłać informacje do określonych jednostek, do których są podłączone
		\item Klasy abstrakcyjne implementowane przez jednostki przetwarzania dostarczają pozwalający na to mechanizm
	\end{itemize}
	\item 5. Propagacja wyników przetwarzania i notyfikacji o ich zmianach
	\begin{itemize}
		\item Programista potrzebuje mechanizmu pozwalającego przedstawić zbiór wartości (zbudowany m.in. na podstawie wyników analizy dźwięku) w ustrukturyzowany, a zarazem rozszerzalny i dynamiczny sposób
		\item System dostarcza zestaw gotowych klas mogących służyć do zbudowania hierarchicznego modelu danych będącego obiektem pierwszej klasy. Przedmiot obserwacji stanowią dane reprezentowane przez konkretne, identyfikowane za pomocą ścieżki podmodele. Zadaniem programisty pozostanie zbudowanie potrzebnej mu struktury z dostarczanych przez framework elementów oraz dostarczenie wartości.
	\end{itemize}
	\item 6. Odtwarzanie dźwięku
	\begin{itemize}
		\item Programista wczytuje plik (1)
		\item Programista tworzy instancję Playera (dostarczany przez system)
		\item Programista przekazuje Playerowi wczytany plik lub podpina go do Sample Enumeratora pliku (co pozwala na odtwarzanie równolegle z przetwarzaniem).
		\item Player (system) udostępnia podstawową kontrolę nad dźwiękiem (stop, pauza).
	\end{itemize}
\end{itemize}
\subsection{Zaimplementowane gotowe elementy drzewa przetwarzania}
\begin{itemize}
	\item 7. Podział danych na kawałki
	\begin{itemize}
		\item Programista chciałby przetwarzać dane w kawałkach o określonej wielkości.
		\item System dostarcza gotowy element drzewa przetwarzania dostarczający tą funkcjonalność.
	\end{itemize}
	\item 8. Dostanie danych po przejściu przez okno Hamminga
	\begin{itemize}
		\item Programista chciałby by jego segmenty danych posiadły periodyczny charakter niezbędny dla pewnych przekształceń operujących z założenia na funkcjach okresowych (np. dyskretna transformata Fouriera)
		\item System dostarcza gotowy element drzewa przetwarzania dostarczający tą funkcjonalność.
	\end{itemize}
	\item 9. Dostanie wyników transformaty Fouriera
	\begin{itemize}
		\item Programista potrzebuje do analizy wyniku dyskretnej transformaty Fouriera.
		\item System dostarcza gotowy element drzewa przetwarzania dostarczający tą funkcjonalność.
	\end{itemize}
	\item 10. Multipleksacja kanałów
	\begin{itemize}
		\item Programista pragnie wybrać pojedynczy kanał (konkretną kolumnę dwuwymiarowej tablicy floatów) wejścia.
		\item System dostarcza gotowy element drzewa przetwarzania dostarczający tą funkcjonalność.
	\end{itemize}
	\item 11. Próbkowanie wejścia
	\begin{itemize}
		\item Programista pożąda próbkować dane wejściowe z zadaną częstotliwością (np. co 20)
		\item System dostarcza gotowy element drzewa przetwarzania dostarczający tą funkcjonalność.
	\end{itemize}
	\item 12. Wyliczenie zakresu fali
	\begin{itemize}
		\item Programista pragnie obliczać minimalną i maksymalną wartość amplitudy fali (zakres wartości pojedynczego wejścia dla każdego kanału).
		\item System dostarcza gotowy element drzewa przetwarzania dostarczający tą funkcjonalność.
	\end{itemize}
	\item 13. Wyliczenie mocy fali
	\begin{itemize}
		\item Programista chciałby poznać moc fali (definiowane jako średnia kwadratowa amplitudy wartości próbek w danym fragmencie) każdego kanału.
		\item System dostarcza gotowy element drzewa przetwarzania dostarczający tą funkcjonalność.
	\end{itemize}
	\item 14. Wyliczenie spektrum mocy fali
	\begin{itemize}
		\item Programista chciałby przeanalizować spektrum mocy fali.
		\item Programista do wyjścia transformaty Fouriera (9) może podłączyć dostarczany przez system gotowy komponent dostarczający tą funkcjonalność.
	\end{itemize}
\end{itemize}



