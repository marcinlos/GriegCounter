\chapter{Aplikacja Androidowa}
\section{Ogólna budowa aplikacji}
U podstawy całej aplikacji stoją dwie aktywności Androidowe. \code{GriegMain} jest ekranem głównym aplikacji, jego celem jest pozwolenie użytkownikowi na podstawową konfigurację oraz wybranie pliku muzycznego do analizy. \code{ResultsActivity} To aktywność robocza, w niej zachodzą wszystkie obliczenia, tam też prezentowane są wyniki. Odpowiadające im pliki .xml odpowiadają za wygląd oraz rozmieszczenie elementów. Dla prostoty użytkowania powiązanie pomiędzy aktywnościami a deklaratywnym opisem wyglądu zapewniane jest przez mechanizm Dependency Injection realizowany przez RoboGuice.

\section{Inicjalizacja}
Przy starcie aplikacji, przed pokazaniem użytkownikowi interfejsu w klasie \code{GriegApplication} inicjalizowane są dwie osobne fabryki drzewa przetwarzania.„Domyślna”, realizowana przez \code{DefaultAndroidBootstrap} tworzy drzewo wykorzystujące wszystkie przygotowane przez nas węzły, natomiast \code{SimpleAndroidBootstrap} dołącza jedynie te elementy które nie wymagają wyświetlania swoich wyników w czasie rzeczywistym (wykresy fali, mocy oraz metadane). Obydwie fabryki gotowe są do pobrania z każdego elementu aplikacji.

\section{GriegMain}
Layout tej aktywności składa się z jednego przycisku oraz checkboxa. Checkbox pełni rolę konfiguracji, pozwala użytkownikowi dynamicznie wybrać którą fabrykę chciałby wykorzystać. Naciśnięcie przycisku natomiast otwiera okno dialogowe w którym znajduje się nasza prosta implementacja eksploratora plików pozwalającego odnaleźć plik do analizy na karcie SD urządzenia. Wykorzystuje on manualną kontrolę kanału alfa by uzyskać płynne przejście przez przeźroczystość, wykorzystuje też kilka ikon folderów dostępnych w internecie na licncji zgodnej z ideami OpenSource. Wybranie pliku powoduje przejęcie kontroli przez aplikację: najpierw do \code{SharedPreferences} zapisywane jest obecne ustawienie użytkownika - będzie ono persystentne pomiędzy uruchomieniami, jak również dostępne z poziomu innej aktywności. Tworzony jest również \code{Intent} przechodzący do \code{ResultsActivity} w którym umieszczamy uchwyt do wybranego pliku, po czym zmieniamy aktywność

\section{ResultsActivity}
Jest to aktywność dziedzicząca po \code{RoboTabActivity}, i składa się z kilku niezależnych od siebie zakładek. Na każdej z nich (poza ostatnią przeznaczoną na metadane) znajduje się jeden lub dwa wykresy wstrzykiwane z widoku z pomocą Guice'a. Inicjalizowane są wszystkie zakładki, po czym usuwane są te które wyświetlać się nie powinny (zależnie od wybranej przez użytkownika opcji). Dalej inicjalizuje się pasek postępu który wskazywać będzie postęp przetwarzania wstępnego. Następnie w oparciu o (odpowiednią) fabrykę manualnie tworzymy elementy Modelu by połączyć logikę z widokiem. Na końcu zaś w osobnym wątku uruchamiany jest dwuetapowy proces przetwarzania.

Gdy przetwarzanie wstępne zostanie zakończone (metadane zostają wczytane) pasek postępu znika i rozpoczyna się analiza właściwa której wynikami są rysowane w czasie rzeczywistym z pomocą klas \code{LineChartView} oraz \code{SpectrumView} wykresy.

\section{Wykresy}
\code{LineChartView} jest widokiem, klasą rysującą wykresy. Źródłem danych dla niej jest Model ,będący pośrednikiem pomiędzy logiką przetwarzania a GUI, przekazywany za pomocą settera do którego widok dołącza się jako Listener słuchający zmian danych. Same dane przekazywane z modelu kolekcje punktów (choć nie wymagane na poziomie składni oczekuje, że będą to pary punktów) z zakresu [-1,1] dla współrzędnej Y (lub [0,1] w wariancie \code{scaleChanged}) oraz [0,1] dla współrzędnej X, stanowiących kompletny zestaw informacji do narysowania na ekranie. Każde uaktualnienie powoduje ponowne odrysowanie całości w oparciu o obecny zestaw punktów. Takie podejście pozwala go stosować zarówno do wykresów zmieniających się w czasie rzeczywistym (takich jak okno fali) których modele przekazują kolejne pełne obrazy do odrysowania, jak i wykresy statyczne (takie jak wykres mocy fali) których modele przysyłają gradualnie rosnące kolekcje punktów których odrysowywanie powoduje efekt powolnego przyrostowego rysowania całości na oczach użytkownika.

Ta kolekcja punktów stanowi model danych który jest rysowany niskopoziomowo bezpośrednio na Canvasie za pomocą linii:
\begin{java}
    public int getScreenX(Point p) {
        return (int) (p.x * viewWidth);
    }
   
    public int getScreenY(Point p){
        float y = 1-p.y;
        if(!scaleChanged)
            y = 0.5f * (1 - p.y);
        return (int) (y * viewHeight);
    }

    private void drawLines(Canvas canvas) {
        for (int i = 1; i < data.size(); ++i) {
            Point a = data.get(i - 1), b = data.get(i);
            canvas.drawLine(getScreenX(a), getScreenY(a), getScreenX(b), getScreenY(b), paint);
        }
    }
\end{java}
Każda linia zbudowana jest z dwóch punktów których współrzędne na bieżąco przeskalowywane są do proporcji ekranu i konwertowane by pasować do układu współrzędnych w którym punkt (0,0) to lewy górny róg ekranu.

\code{SpectrumView} działa na podobnej zasadzie, stworzony został jednak do rysowania wykresu słupkowego reprezentującego spektrum fali - czyli natężenie poszczególnych zakresów częstotliwości w skali logarytmicznej na obu współrzędnych. By móc takowy uzyskać wyniki FFT sygnału dźwiękowego muszą najpierw zostać przetworzone przez \code{SpectrumBinsCalculator} który na podstawie minimalnej częstotliwości i liczby słupków zwraca oczekiwaną tablicę wysokości. Szerokość słupków dobierana jest tak by wypełniały całą dostępną przestrzeń.
