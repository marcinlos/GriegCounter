\chapter{Cel pracy i wizja produktu}

\section{Wprowadzenie}
Wiele uwagi dzisiejszego świata poświęcone jest muzyce. Jeden z najbardziej dochodowych filarów przemysłu rozrywkowego, nieodłączny element naszego życia. Pomimo tego narzędzia do analizy muzyki są jednak w zarodku. Na palcach jednej ręki można policzyć aplikacje które są nam w stanie powiedzieć coś o pliku dźwiękowym. Dysponując telefonem z systemem Android praktycznie jedyne co jesteśmy w stanie dowiedzieć się to tytuł piosenki - pod warunkiem że zapłacimy za odpowiednią aplikację, mamy dostęp do internetu a aranżacja jest bliska oryginalnej. Co natomiast z ludźmi którzy chcieliby wiedzieć więcej? Zobaczyć falę dźwiękową, poznać tempo, być może rozpoznać gatunek bez konieczności odsłuchiwania utworu? Tą lukę postanowiliśmy wypełnić
\section{Cel i wizja}
Celem naszej pracy było zaprojektowanie i zaimplementowanie mobilnej aplikacji będącej w stanie robić właśnie te rzeczy - wizualizować, wyciągać informacje, analizować, informować o technicznych aspektach muzyki. Wszystko to zaś zamknięte w okowach systemu operacyjnego Android. Podczas pierwszych faz developmentu odkryliśmy jednak dość nieoczekiwany fakt - Android nie implementuje wszystkich bibliotek Javy. W tym tej, która byłaby nam najbardziej potrzebna: java.sound

By temu zaradzić cel naszej musiał zostać lekko rozszerzony. Ze względu na pozbawienie nas podstawowych niskopoziomowych narzędzi dostępu do pliku muzycznego jak i nadziei na wykorzystanie istniejących już bibliotek - postanowiliśmy zaprojektować własną. Tak narodził się Grieg, framework umożliwiający odczyt i analizę plików dźwiękowych. W jego ramach zamknięta zostałaby logika analizy, aplikacja zaś swój cel spełniałaby poprzez odpowiednie użycie oraz prezentację logiki zawartej wewnątrz Frameworku.

\section{Analiza ryzyka}
Tak postawiony cel niósł ze sobą spore ryzyko:
\\
\emph{Stopień skompikowania} - żaden z nas nie miał wcześniej do czynienia z pracą nad plikami dźwiękowymi, projektowaniu frameworków a nawet doświadczenia w pisaniu aplikacji na Androida. Na naszą korzyść działała tutaj jedynie ogromna wiedza programistyczna jednego z członków zespołu \emph{Ryzyko: 8/10}
\\
\emph{Praca w zespole} - charakter członków zespołu mógł okazać się niekompatybilny. Problemy z podziałem pracy i delegacją obowiązków, niepewny zakres odpowiedzialności, problemy związane z zarządzaniem czasem. \emph{Ryzyko: 7/10} 
 
\chapter{Zakres funkcjonalności}
\section{Aktorzy}
Jako że wynikiem całego projektu będą de facto dwa osobne produkty, wyróżnić możemy dwóch różnych aktorów będących użytkownikami końcowymi. Framework skierowany jest do programistów, będą go chcieli wykorzystać w budowaniu własnych aplikacji, interesować ich będzie kod oraz sposób jego użycia. Aplikacja zaś skierowana jest do użytkownika końcowego, nie posiadającego żadnej wiedzy na temat programowania, który po prostu chce dowiedzieć się kilku rzeczy na temat posiadanego przez niego pliku dźwiękowego.
\section{Programista}
\IncludeUML{n_programista}{Główne przypadki użycia programisty}
\IncludeUML{n_tree}{Przypadki użycia drzewa przetwarzania}
W skład ekstrakcji danych wchodzą poszczególne konkretne informacje jakie można wyciągnąć z pliku muzycznego, po dalsze rozwinięcie odsyłamy do odpowiedniej części dokumentacji.
\section{Użytkownik}
\IncludeUML{n_userMain}{Przypadki użycia użytkownika}
W skład wyświetlania wyników analizy wchodzą konkretne wykresy uzyskiwane w wyniku przetwarzania pliku, po więcej informacji odsyłamy do odpowiedniej części dokumentacji
\chapter{Architektura systemu}

Projekt składa się z dwóch części: frameworku do obsługi dźwięku, oraz aplikacji Androidowej,
stanowiącej ilustrację jego możliwości i przykład użycia. Framework spełnia kilka zadań, niezbędnych
do pracy z plikami dźwiękowymi -- odpowiedzialny jest za ich wczytywanie, konwersję różnych formatów
dźwięku na ten używany wewnętrznie przez framework, oferuje także elastyczny model jego
przetwarzania oraz umożliwia jego odtwarzanie. W jego skład wchodzi kilka modułów:

\IncludeUML{modules}{Wysokopoziomowa struktura projektu}
\begin{itemize}

\item Moduł wczytywania plików - łatwo rozszerzalny zbudowany na pluginach mechanizm wczytania i konwersji plików dźwiękowych do formatu zmiennoprzecinkowego PCM. Implementacja pozwala na wczytywanie plików .mp3, .wav oraz .ogg.

\item Moduł odczytu metadanych - moduł pozyskujący informacje o utworze i reprezentującym go pliku nie stanowiące bezpośrednio danych audio takich jak autor czy wielkość pliku.

\item Moduł przetwarzania dźwięku - udostępnia infrastrukturę pozwalającą w dowolny sposób analizować i
przekształcać wejściowy strumień danych audio. W tym celu wykorzystuje koncept węzłów przetwarzania
-- autonomicznych jednostek, realizujących proste, atomiczne przekształcenia strumienia wejściowego
i generujących strumień wyjściowy, połączonych ze sobą wejściami/wyjściami w strukturę drzewiastą. Konstrukcja ta posiada pożądane przez nas cechy takie jak modularność, komponowalność oraz przede wszystkim prosty i efektywny przepływ danych. Framework udostępnia też kilkanaście gotowych jednostek przetwarzania, realizujących typowe, niezbyt skomplikowane przekształcenia, np. obliczanie mocy fali czy FFT.

\item Moduł propagacji danych -  oferuje możliwość względnie prostego i wygodnego przekazania danych z
logicznej części aplikacji do interfejsu użytkownika. Cel ten zrealizowany został poprzez wprowadzenie hierarchicznego \emph{modelu} danych i wzorca \emph{Observer}. Framework udostępnia klasy
łączące go z modułem przetwarzania, co pozwala stworzyć np. wizualizację transformaty Fouriera
przetwarzanego sygnału w czasie rzeczywistym.

\item Moduł odtwarzania - dostarcza możliwości odtworzenia danych dźwiękowych, odczytanych z pliku bądź
przekształconych. Jest rozszerzalny -- dostępne implementacje stosownych interfejsów są dynamicznie
ładowane podczas uruchomienia na podstawie zawartości CLASSPATH-a. Jego funkcjonalność zbudowana
jest na niewielkim zestawie prostych operacji, dzięk czemu dostarczenie implementacji na nową,
niewspieraną platformę nie wymaga dużego wysiłku. Dostępna jest implementacja oparta o
\textit{JavaSound}.
\end{itemize}

Całość dopełnia aplikacja Androidowa, której budowa została maksymalnie uproszczona by lepiej spełniał swoje zadanie prezentacji budowy aplikacji w oparciu o Framework bez niepotrzebnych dodatków. Składa się ona z dwóch aktywności - Pierwszej, głównej, na którą składa się między innymi prosta implementacja eksploratora plików, oraz druga oparta na zakładkach przystosowana do wyświetlania danych przetwarzania.

\chapter{Organizacja pracy}
Planem na organizację pracy początkowo była idea iteracji, przyrostowego modelu którego każdy cykl kończy się gotowym do zaprezentowania klientowi produktem. Po przerzuceniu ciężaru prac na budowę Frameworka okazało się to jednak w dużej mierze niemożliwe ze względu na jego zawiłą i powiązaną nawzajem budowę uniemożliwiającą działanie na wynikach cząstkowych. Stąd też jedna z iteracji zdecydowanie góruje długością nad pozostałymi ponieważ składa się na nią całość implementacji Frameworku od zera do stanu w którym był on używalny a API przyjęło ostateczną formę.

\section{Struktura zespołu}
Nasz zespół pracował w ścisłej hierarchii związanej z poziomem umiejętności. Kierownictwo przejął \emph{Marcin Łoś} - to on odpowiada za architekturę frameworku, on też zarządzał projektem od strony programistycznej delegując zadania gdzie uznał to za stosowne. \emph{Michał Torba} jako główną odpowiedzialność przyjął aplikację Androidową oraz dokumentację projektu pełniąc jedynie pomocniczą funkcję w pozostałych aspektach prac.
\section{Skrót harmonogramu}
%TODO gdy skoncze procesowa
\section{Testy jednostkowe i akceptacyjne}

Do testów jednostkowych wykorzystywany był framework JUnit4 oraz Mockito, sporadycznie także
PowerMock. Nie cały kod został pokryty w pełni testami jednostkowymi; pokrycie wynosi ok. 30\% (do
jego określenia wykorzystane zostało narzędzie EclEmma). Testowane były głównie te fragmenty,
których poprawność jest kwestią delikatną i łatwą do naruszenia, m. in. systemy dynamicznego
ładowania dodatkowej funkcjonalności wraz z parserami konfiguracji, wykorzystywane przez nie klasy
pomocnicze do skanowania CLASSPATH-a i obsługa refleksji, jak również API modeli i klasy
odpowiedzialne za konfigurację. Brak testów jednostkowych dla rdzenia systemu spowodowany jest jego
stabilnością -- po stworzeniu go na samym początku prac implementacyjnych nie był zmieniany.


Aplikacja Androidowa, ze względu na korzystanie z zaimplementowanych i przetestowanych już rozwiązań Frameworku poddana została jedynie testom akceptacyjnym - uruchamiana była na kilku różnych emulatorach reprezentujących różne urządzenia oparte o system Android, oraz wykorzystana do przeanalizowania różnych plików.
\section{Narzędzia}
Jako repozytorium kodu wykorzystaliśmy popularny Github. Do szybkiego tworzenia bieżącej dokumentacji projektowej korzystaliśmy z systemu Confluence udostępnianego przez Katedrę Informatyki. Specjalne podziękowanie należy się też Google Hangouts, głównej platformy komunikacji wewnątrz zespołu.

Środowiskiem budowy był Eclipse skonfigurowany pod współpracę z Android SDK. Funkcję emulatorów pełniła maszyna wirtualna wbudowana w SDK, Bluestacks będący szybką Windowsową aplikacją emulującą Androida, oraz GenyMotion czyli maszyna wirtualna systemu Android działająca pod VirtualBoxem którą można wykorzystywać dla celów niekomercyjnych do testowania aplikacji.

\chapter{Wyniki projektu}
\section{Efekty i ocena przydatności}
W ramach przyjętych założeń uzyskaliśmy zadowalające wyniki. Framework spełnia wszystkie założenia techniczne przed nim postawione, jest stosunkowo prosty w użytkowaniu i stosownie udokumentowany. O ile gotowe jednostki przetwarzania mogą okazać się niewystarczające dla programistów pragnących w oparciu tylko o nie zbudować komercyjną aplikację, to framework jest doskonałym punktem wyjścia dla programistów pragnących napisać własne elementy analizy. Aplikacja androidowa dobrze demonstruje sposób wykorzystania Frameworku dając punkt wyjścia dla przyszłych bardziej złożonych implementacji.

\section{Ograniczenia i kierunki dalszych prac}
Największym ograniczeniem jest liczba gotowych komponentów - te gotowe nie wzbudzą zachwytu w entuzjastach analizy plików muzycznych, reprezentują jedynie absolutne podstawy. Również aplikacja Androidowa ostatecznie może wymagać solidnych przeróbek jeśli nie całkowitego przepisania przez ludzi lepiej zaznajomionych z zawiłościami środowiska Androidowego.

\nocite{*}
\bibliography{przewodnik}
