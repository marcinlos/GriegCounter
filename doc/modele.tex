\chapter{Model}
\section{Moduł propagacji danych}

Jakkolwiek kwestia komunikacji z interfejsem użytkownika nie jest ściśle związana z tematyką
frameworku, jednak jako że jest on tworzony z myślą o aplikacji, konieczne było zastosowanie
jakiegoś rozwiązania tego problemu. 

Głównym zadaniem projektowym jest wyraźne odseparowanie kodu generującego zdarzenia i dane --
,,logiki'' systemu, jego wewnętrznych procesów i struktury -- od części zajmującej się wizualizacją
i interakcją z uzytkownikiem. Istnieje wiele wzorców architektonicznych pozwalających na różne
sposoby, w mniejszym lub większym stopniu ten cel osiągnąć, na czele z powszechie znanym, jednym z
najwcześniejszych -- MVC. Rozwiązanie zastosowane we frameworku i aplikacji zbudowanej na jego
podstawie opiera się bardziej na wariancie MVVM (\textit{Model-View-ViewModel}), znanym choćby z
technologii .NET-owych (XAML, data-binding), szczególnie komunikacja w stronę logika $\to$ GUI.
Logika aplikacji udostępnia dane w postaci jednego obiektu -- struktury hierarchicznej,
przypominającej w pewnym stopniu system plików. Poszczególne dane są adresowane przy pomocy ścieżek.
Elementy GUI mogą zarejestrować listenery, przez które powiadamiane będą o zmianach obserwowanych
wartości.


\subsection{Implementacja}

\IncludeUML{model_structure}{Klasy składające się na implementację modułu propagacji danych}

Najistotniejszą rolę pełni interfejs \code{Model} i jego implementacje. Zawiera on operacje, które
podzielić można na kilka grup, względem zakresu odpowiedzialności:

\begin{itemize}

  \item zarządzanie daną -- funkcje \code{getData}, \code{getDataType} i \code{update} pozwalają
    pobrać\slash zmienić wartość danej reprezentowanej przez model

  \item wyszukiwanie dzieci -- kolejna grupa funkcji pozwala pobierać informacje o modelach leżących
    niżej w hierarchii

  \item zarządzanie listenerami -- pozwalają dołączać\slash usuwać listenerów zarejestrowanych w tym
    modelu, bądź w pewnym jego podmodelu, wyspecyfikowanym przez względną ścieżkę

\end{itemize}

Częściową implementację interfejsu \code{Model} stanowi klasa abstrakcjna \code{AbstractModel}.
Dostarcza ona implementacji wszystkich wymienionych wyżej aspektów poza zarządzaniem dziećmi -- te
zawarte są w dwóch konkretnych jej podklasach, \code{SimpleModel} i \code{CompositeModel}.
\code{SimpleModel} to model bez dzieci, natomiast \code{CompositeModel} wspiera tworzenie modeli z
dowolną ilością dzieci oraz zarządzanie nimi, poprzez metody \code{addModel} i \code{removeModel}.

Zmiany wartości modelu komunikowane są poprzez minimalistyczny interfejs \code{Listener},
zawierający tylko jedną metodę -- \code{udpate}, wywoływaną z nową wartością przy zmianie wartości
modelu.

Ścieżki w modelu reprezentuje klasa \code{Path}. Składa się ona z ciągu znaków reprezentującego
ścieżkę, oraz listy jej \emph{komponentów} -- alfanumerycznych części oddzielonych znakem kropki
(.), będącymi nazwami kolejnych podmodeli na drodze prowadzącej do tego reprezentowanego przez tę
ścieżkę.

\IncludeUML[0.4]{model_usage}
{Przykład użycia modeli w aplikacji -- dane z drzewa przetwarzania i innych źródeł agregowane są w
drzewie modelu, z którego korzysta GUI}

Powyższy diagram obrazuje miejsce i rolę opisanego w tej sekcji modułu w aplikacji. Modele mogą
służyć do przekazywania wartości reprezentujących wyniki przetwarzania, ale ich użycie do tego się
nie ogranicza. Przykład stanowi użycie drzewa modeli do przekazania GUI informacji o rozszerzeniach
plików, dla których istnieją zarejestrowane parsery, potrzebnej do poprawnego skonstruowania okna
wyboru pliku tak, aby domyślnie wyświetlane były jedynie pliki obsługiwanych typów (przykład
pochodzi z GUI testowego zbudowanego z użyciem Swinga). Podobnie, informacje o postępie procesu
ekstrakcji metadanych opakowane są w stosowny model, uaktualniany poprzez listenera (mechanizmy
opisane w sekcji dot. metadanych), który to model używany jest do wyświetlenia i uaktualniania paska
postępu.

