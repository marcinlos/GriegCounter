
\chapter{Wstep}

Projekt podzielić można na dwie odrębne, choć związane części. Pierwsza to ogólnego zastosowania
framework do odczytu, analizy i odtwarzania dźwieku, zaś druga to aplikacja Androidowa stanowiąca
ilustrację jego możliwości i przykład użycia.

Framework spełnia kilka zadań. Umożliwia wczytywanie plików audio w kilku wspieranych bazowo
formatach (wav, mp3, ogg) i daje dostęp do zawartych w nich danych w spójnym formacie, niezależnym
od reprezentacji dźwięku w pliku wejściowym. Pozwala w prosty sposób rozszerzać tę funkcjonalność na
nowe formaty, poprzez dodanie własnych implementacji odpowiedniego interfejsu. Oferuje elastyczny
model przetwarzania dźwięku, wzorowany na konstrukcjach zastosowanych w innych tego typu
bibliotekach.

Framework podzielony jest na kilka współpracujących ze sobą modułów. Ogólną, wysokopoziomową jego
struktura przedstawiona została na poniższym diagramie.

\IncludeUML{modules}{Struktura frameworka i powiąznaia z używającą go aplikacją}

Moduł wczytywania plików zajmuje się odczytywaniem plików z danymi audio, dekodowaniem ich i
konwersją otrzymanych w ten sposób danych do spójnego wewnętrznego formatu frameworku. Do jego zadań
należy zarządzanie podmodułami obsługi konkretnych formatów plików.

Moduł odczytu metadanych służy do pozyskiwania informacji o pliku nie będących bezpośrednio danymi
audio (tytuł, autora itd). 

Moduł przetwarzania dźwięku udostępnia infrastrukturę pozwalającą w dowolny sposób analizować i
przekształcać wejściowy strumień danych audio. W jego skład wchodzą interfejsy i klasy pomocnicze
budujące rdzeń frameworka, a także zestaw gotowych podstawowych przekształceń i obliczeń.

Moduł propagacji danych oferuje możliwość względnie prostego i wygodnego przekazywania danych z
logicznej części aplikacji do interfejsu użytkownika. Framework udostępnia klasy łączące go z
modułem przetwarzania, co pozwala stworzyć np. wizualizację transformaty Fouriera przetwarzanego
sygnału w czasie rzeczywistym.

Moduł odtwarzania dostarcza możliwość odtworzenia danych dźwiękowych, odczytanych z pliku bądź
przekształconych. Jest rozszerzalny -- jego funkcjonalność zbudowana jest na niewielkim zestawie
prostych operacji, dzięki czemu dostarczenie implementacji na nową, niewspieraną platformę nie
wymaga dużego wysiłku.

Powyższe opisy mają na celu jedynie pobieżne ukazanie ogólnej struktury frameworku. Kolejne sekcje
dokumentacji technicznej opisują szczegółowo implementację poszczególnych modułów.
